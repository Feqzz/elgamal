\documentclass{article}
\usepackage[parfill]{parskip}

\author{Stian Onarheim}
\title{RFMA310-1 20H Diskret matematikk Hjemmeeksamen}

\begin{document}
\maketitle
\newpage
\tableofcontents
\newpage

\section{The Elgamal Encryption Algorithm}
Helloo

\section{The source code}
I have implemented Elgamal in python as it supports enormous numbers within its default libraries. Before encrypting the plaintext, I convert its characters to ASCII values and concatenates them together.\\

The message to be encrypted has to be an element of the cyclic group $G$, creating a limit to the message's length. To support longer messages, the message is divided into blocks smaller than $G$'s order. As the ASCII values varies from one - three digits, zeros are appended at the beginning to make every value the same length. This is needed to make decryption easier. \ref{blocks}.
\end{document}
